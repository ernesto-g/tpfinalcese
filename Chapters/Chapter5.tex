% Chapter Template

\chapter{Conclusiones} % Main chapter title

\label{Chapter5} % Change X to a consecutive number; for referencing this chapter elsewhere, use \ref{ChapterX}

%En este capítulo se detallan las conclusiones del presente trabajo y los pasos a seguir.

%----------------------------------------------------------------------------------------

%----------------------------------------------------------------------------------------
%	SECTION 1
%----------------------------------------------------------------------------------------

\section{Conclusiones generales }

En el presente trabajo se ha logrado implementar un conjunto de herramientas que permiten a estudiantes  de educación secundaria y universitaria aprender programación sobre sistemas embebidos, así como la enseñanza de programación orientada a objetos, cumpliendo con la totalidad de los requerimientos planteados. Esto fue posible gracias a la decisión de utilizar el lenguaje de programación Python y la plataforma de hardware EDU-CIAA-NXP, proveyendo al alumno de un entorno de trabajo simple y bien documentado eliminando las barreras mencionadas en el capítulo \ref{Chapter1} que dificultan el aprendizaje en el ámbito de la informática y los sistemas embebidos.

A lo largo del desarrollo de este trabajo el autor ha aplicado los conocimientos adquiridos en diversas materias de la carrera, mayormente de las que poseen contenidos relacionados con programación de microcontroladores, sin embargo también se ha tomado una base importante de la materia de ingeniería de software, la cual influyó fuertemente en las tareas de verificación y validación que se realizaron para asegurar el correcto funcionamiento de las partes implementadas.

Como casos de éxito de este trabajo, pueden mencionarse el dictado de una clase en el curso \textit{Paquete Tecnológico del Proyecto CIAA}\footnote{URL: http://www.proyecto-ciaa.com.ar/devwiki/doku.php?id=educacion:cursos:cursos\_programacion\_ciaa} en el marco de los Cursos Abiertos de Programación de Sistemas Embebidos, así como en una clase acerca de programación orientada a objetos sobre sistemas embebidos de la \textit{CESE}\footnote{CESE: Carrera de Especialización en Sistemas Embebidos.}. En ambos casos, los alumnos utilizaron una versión preliminar de este trabajo y manejando los leds y pulsadores de la placa pudieron realizar un gran número de prácticas.

En base a lo descripto anteriormente, se puede asegurar que los objetivos del trabajo se han cumplido ampliamente y el desarrollo del mismo ha favorecido en gran medida a la formación profesional del autor.


%----------------------------------------------------------------------------------------
%	SECTION 2
%----------------------------------------------------------------------------------------
\section{Próximos pasos}

Como se menciona en la sección \ref{sec:micropython} este trabajo partió de la utilización de una capa de soporte de hardware desarrollada en 2015 la cual se reescribió y mejoró utilizando técnicas de ingeniería de software (las cuales no implementaba) para los periféricos mencionados en los requerimientos de este trabajo (sección \ref{sec:req}). Existe otro grupo de periféricos para los cuales se debe hacer este mismo trabajo de refactoring e implementación de tests, los cuales se detallan a continuación:

\begin{enumerate}
	\item  Interrupciones.
	\item  PWM.
	\item  Keyboard y LCD (Poncho UI). 
	\item  SPI e I2C (modo master).
	\item  RTC.
\end{enumerate}

También existen bibliotecas programadas exclusivamente en Python como el soporte de \textit{Modbus}\footnote{Modbus:Modbus un protocolo de comunicaciones situado en el nivel 7 del Modelo OSI, basado en la arquitectura maestro/esclavo (RTU) o cliente/servidor (TCP/IP), diseñado en 1979.} y operaciones con fecha y hora las cuales tampoco fueron implementadas siguiendo requerimientos ni validadas mediante ningún tipo de test. Y por último existen periféricos para los que no hay soporte y se debería comenzar de cero, como USB, CAN, Ethernet, los modos slave de los buses I2C y SPI y el core del cortex M0.

Con respecto al IDE, existe una nueva versión beta la cual posee autocompletado y una ventana con tips de ayuda, ambas características no han sido desarrolladas con requerimientos ni validadas de ninguna manera.

También existe una versión preliminar de un emulador de la placa EDU-CIAA-NXP, creado por el autor de esta trabajo, el cual soporta ser lanzado desde otro programa (por ejemplo el IDE desarrollado) y simula el script de Python programado, por el momento solo soporta el uso de leds y pulsadores, y no simula el resto de los periféricos, pero la continuación de este proyecto a una versión estable garantiza una versión del conjunto de herramientas desarrollado la cual puede correr en una PC en su totalidad, eliminando la necesidad de que exista una placa por cada alumno o grupo de alumnos y bajando de esta forma los costos requeridos para utilizar esta herramienta en ambientes educativos.

Con respecto a la ayuda y documentación, el autor de este trabajo tiene la idea de desarrollar video-tutoriales en donde se muestre la ventana del IDE mientras se programa y se explica paso por paso lo programado, con ayuda de gráficos u otras herramientas en el caso de requerirse. Estos videos también podrían estar divididos en diferentes niveles de complejidad como se hizo con el repositorio de ejemplos.

Por último existe la idea de generar una release del firmware en su versión binaria, para poder ser programada en la placa sin necesidad de instalar en la PC las herramientas de compilación, esto se logra puenteando el jumper JP5 de la placa y ejecutando programas como \textit{lpc21isp}\footnote{lpc21isp:https://sourceforge.net/projects/lpc21isp} o \textit{FlashMagic}\footnote{Flashmagic:http://www.flashmagictool.com} los cuales permiten grabar el microcontrolador por medio de un puerto serial. Si bien es verdad que en el caso de una entidad educativa solo el profesor podría instalarse las herramientas de compilación y grabar las placas con el firmware para los alumnos, es una buena idea disponer de una manera simple de grabar la EDU-CIAA-NXP para que principiantes y autodidactas que han adquirido la placa puedan comenzar a programar sin inconvenientes.


 



